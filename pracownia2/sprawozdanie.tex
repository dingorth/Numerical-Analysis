\documentclass{article}
\usepackage[utf8]{inputenc}
\usepackage{polski}

\usepackage{bbm}
\usepackage{graphicx}    
\usepackage{caption}
\usepackage{subcaption}
\usepackage{epstopdf}
\usepackage{amsmath,amssymb,amsfonts,amsthm,mathtools}
\usepackage{hyperref}
\usepackage{url}
\usepackage{comment}
\usepackage[section]{placeins}
\newtheorem{defi}{Definicja}
\newtheorem{twr}{Twierdzenie}


\author{Jan Mazur 281141}
\date{Wrocław, \today}
\title{\textbf{Naturalne funkcje sklejane III stopnia} \\ Sprawozdanie do zadania P.2.9	}

\begin{document}
\maketitle

\section{Wstęp}

"Interpolacja to metoda numeryczna polegająca na wyznaczaniu w danym przedziale tzw. funkcji interpolacyjnej, która przyjmuje w nim z góry zadane wartości, w ustalonych punktach nazywanych węzłami." \cite{wikipedia_pl}
Najczęściej stosowana jest interpolacja wielomianami, ponieważ  mają one sporo przydatnych własności, co imlikuje istnienie wielu narzędzi matematycznych do ich analizy. Jednakże nie zawsze jest to dobra metoda.
Zastosować wtedy można inny sposób interpolacji - interpolację funkcjami sklejanymi.
Przedstawię obię metody, skupiając się jednak na interpolacjami funkcjami sklejanymi. Porównam ich błędy w stosunku do funkcji interpolowanej.

Wszelkie testy i obliczenia zostały wykonane w środowisku Jupyter Notebook przy użyciu języka programowania Julia w wersji 0.5.0.
Znaleźć można je w dołączonym pliku program.ipynb.\\\\
\noindent
Zanim przystąpimy do obliczeń ustalmy najpierw funkcję błędu:

\begin{defi}
	$E_N^{(n)} := \max_{x \in D_N}|f(x)-s(x)))|$ \\
	\\
	Gdzie s jest funkcją interpolacyjną w n+1 parami różnych węzłach z przedziału $[a,b]$, f funkcją interpolowaną a $D_N$ zbiorem parami różnych równoodległych punktów z przedziału $[a,b]$.
\end{defi}



\section{Interpolacja wielomianowa}
Interpolacja wielomianowa polega na znalezieniu wielomianu $n$-tego stopnia, który w $n+1$ węzłach będzie miał te same wartości co funkcja interpolowana. Rozwiązaniem tego problemu jest interpolacja Lagrange'a. Rozważmy wielomiany interpolacyjne w postaci Newtona dla następujących funkcji:

\begin{equation}\label{fun1}
	f(x) = sin(x), \textnormal{ } x \in [0,\pi]
\end{equation}
\begin{equation}\label{fun2}
	f(x) = e^x, \textnormal{ } x \in [0,4]
\end{equation}
\begin{equation}\label{fun3}
	f(x) = (x^{2}+1)^{-1}, \textnormal{ } x \in [-5,5]
\end{equation}
\begin{equation}\label{fun4}
	f(x) = x/(x^{2} + \frac{1}{4}), \textnormal{ } x \in [-\pi,\pi]
\end{equation}


W przypadku funkcji \eqref{fun1} i \eqref{fun2} błąd wielomianu interpolacyjnego jest niewielki. Natomiast przy funkcjach \eqref{fun3} i \eqref{fun4} zaobserwować można tzw. efekt Runge'go\cite{runge}. Polega on na tym, że przy zwiększaniu ilości węzłów uzyskujemy coraz większy błąd na końcach przedziałów interpolacji. Rozwiązaniem tego problemu może być interpolacja funkcjami sklejanymi.


\section{Funkcje sklejane}


\begin{defi}
	\textbf{Funkcją sklejaną} S(x) stopnia s, określoną na przedziale $[a,b]$ nazywamy dowolną funkcję spełniającą warunki:
	
	\begin{itemize}
		\item w każdym przedziale $[t_i,t_{i+1}]$, gdzie $a = t_0 < t_i <...<t_n = b$, S jest wielomianem stopnia co najwyżej s.
		\item S oraz jej pochodne rzędu 1,2,...,s-1 są ciągłe na całym przedziale $[a,b]$.
	\end{itemize}
\end{defi}

\noindent Aby ujednoznacznić istnienie takiej funkcji n-tego stopnia wyznaczanej przez $n+1$ punktów $t_i$ definiuje się dodatkowe warunki:

\begin{defi}
	Funkcję sklejaną S(x) stopnia s, określoną na przedziale $[a,b]$
	nazywamy \textbf{naturalną} gdy:
	$S^{(s-1)}(a) = S^{(s-1)}(b) = 0$.
\end{defi}

Rozważmy interpolacje naturalną funkcją sklejaną III stopnia gdzie $t_i$ będą węzłami interpolacji.

\section{Interpolacja naturalną funkcją sklejaną III stopnia}

\begin{defi}
	Naturalna funkcja sklejana III S, stopnia s, interpolująca funkcję f w punktach $a = x_0 < x_i <...<x_n = b$ $\in [a,b]$ spełnia warunki:
	
	\begin{itemize}
		\item w każdym przedziale $[x_i,x_{i+1}]$ S jest wielomianem stopnia co najwyżej s.
		\item Dla wszystkich $x_i$ $S(x_i) = f(x_i)$.
	\end{itemize}
	
\end{defi}

\noindent Wprowadźmy oznaczenie:

\begin{defi}
	$M_k := S''(x_k)$
\end{defi}

\begin{twr}
	Wielkośći $M_k$ spełniają układ równań liniowych:\\
	$\lambda_k M_{k-1} + 2M_k + (1-\lambda_k) M_{k+1} = 6f[x_{k-1},x_k,x_{k+1}]$ \hfill $(k=0,1,...,n)$\\
	gdzie: $\lambda_k := h_k / ( h_k + h_{k+1})$, oraz $h_k = x_k - x_{k-1}$
\end{twr}
	
Układ równań z powyższego twierdzenia w postaci macierzowej ma trójprzekątniową macierz z dominującą przekątną. Można go więc rozwiązać w czasie liniowym.

\subsection{Algorytm rozwiązywania układu równań ze względu na $M_k$}
\begin{verbatim}
	algorytm
\end{verbatim}

Naturalna funkcja sklejana III stopnia interpolująca funkcję f w przedziale $[a,b]$ i punktach $a = x_0 < x_i <...<x_n = b$ dana jest wzorem:

\begin{equation}
	S(x) = h_{k}^{-1}[]
\end{equation}




\section{Testy}
Wykresiki i liczenie błędów.

Wybór punktów interpolacyjnych.
	
	
\section{Wnioski}
Jeśli zwykła interpolacja bardzo odstaje w niektórych miejscach to lepiej interpolować splinem.

Czy punkty równoodległe?


\begin{thebibliography}{9}
	\itemsep2pt
			
	\bibitem{kincaid} David Kincaid, Ward Cheney - "Analiza Numeryczna"
	
	\bibitem{prezentacja} \url{https://www.math.ntnu.no/emner/TMA4215/2008h/cubicsplines.pdf}
	
	\bibitem{wolfram_mathworld} Weisstein, Eric W. "Cubic Spline." From MathWorld--A Wolfram Web Resource. \url{http://mathworld.wolfram.com/CubicSpline.html}
	
	\bibitem{wiki} \url{https://en.wikipedia.org/wiki/Spline_interpolation}
	
	\bibitem{wikipedia_pl}
	\url{https://pl.wikipedia.org/wiki/Interpolacja_(matematyka)}
	
	\bibitem{runge}
	\url{https://en.wikipedia.org/wiki/Runge%27s_phenomenon}
	 
\end{thebibliography}

\end{document}