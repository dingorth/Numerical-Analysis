\documentclass{article}
\usepackage[utf8]{inputenc}
\usepackage{polski}

\usepackage{bbm}
\usepackage{graphicx}    
\usepackage{caption}
\usepackage{subcaption}
\usepackage{epstopdf}
\usepackage{amsmath,amssymb,amsfonts,amsthm,mathtools}
\usepackage{hyperref}
\usepackage{url}
\usepackage{alltt}
\usepackage{comment}
\usepackage[section]{placeins}
\graphicspath {{plots/}}
\newtheorem{defi}{Definicja}
\newtheorem{twr}{Twierdzenie}


\author{Jan Mazur 281141}
\date{Wrocław, \today}
\title{\textbf{Naturalne funkcje sklejane III stopnia} \\ Sprawozdanie do zadania P.2.9	}

\begin{document}
\maketitle

\section{Wstęp}

Interpolacja to proces polegający na wyznaczaniu w danym przedziale tzw. funkcji interpolacyjnej, która przyjmuje w nim z góry zadane wartości, w ustalonych punktach nazywanych węzłami. \cite{wikipedia_pl}
Często stosowana jest interpolacja wielomianowa, ponieważ  funkcje te mają sporo przydatnych własności, co implikuje istnienie wielu narzędzi matematycznych do ich analizy. Jednakże nie zawsze jest to dobra metoda.
Zastosować można wtedy inny sposób interpolacji - interpolację funkcjami sklejanymi.

Przedstawię obie metody, skupiając się jednak na interpolacji funkcjami sklejanymi. Porównam ich błędy w stosunku do funkcji interpolowanej.

Wszelkie testy i obliczenia zostały wykonane w środowisku Jupyter Notebook przy użyciu języka programowania \textbf{Julia} w wersji \textbf{0.5.0}, w arytmetyce \textbf{BigFloat} z prezycją 1000.
Znaleźć można je w dołączonym do sprawozdania pliku \textbf{program.ipynb}.\\\\

\section{Interpolacja wielomianowa}
Interpolacja wielomianowa polega na znalezieniu wielomianu $n$-tego stopnia, który w $n+1$ węzłach będzie miał te same wartości co funkcja interpolowana. Rozważmy wielomiany interpolacyjne w postaci Newtona dla następujących funkcji:

\begin{equation}\label{fun1}
	f(x) = sin(x), \textnormal{ } x \in [0,\pi]
\end{equation}
\begin{equation}\label{fun2}
	f(x) = e^x, \textnormal{ } x \in [0,4]
\end{equation}
\begin{equation}\label{fun3}
	f(x) = (x^{2}+1)^{-1}, \textnormal{ } x \in [-5,5]
\end{equation}
\begin{equation}\label{fun4}
	f(x) = x/(x^{2} + \frac{1}{4}), \textnormal{ } x \in [-\pi,\pi]
\end{equation}

Przed przystąpieniem do obliczeń ustalmy najpierw funkcję błędu:
\begin{defi}\label{Error}
	$E_N^{(n)} := \max_{x \in D_N}|f(x)-s(x)))|$ \\
	\\
	Gdzie s jest funkcją interpolacyjną w n+1 parami różnych węzłach z przedziału $[a,b]$, f funkcją interpolowaną a $D_N$ zbiorem parami różnych równoodległych punktów z przedziału $[a,b]$.
\end{defi}

\renewcommand{\arraystretch}{1.5}  
\begin{center}
	\begin{tabular}{||c||c|c|c|c||} \hline
		\multicolumn{5}{||c||}{Tabela błędów dla N = 1000} \\ \hline
		Węzły równoodległe 	& \multicolumn{4}{|c||}{Funkcje} \\ \cline{2-5}
				& $\sin x$ & $e^x$ & $(x^{2}+1)^{-1}$ & $x/(x^{2} + \frac{1}{4})$ \\ \hline					
		3 		& 5.60096e-02 &  6.32501e+00 &  6.46229e-01 & 9.51746e-01 \\ \hline
		5 		& 1.81210e-03 &  3.38690e-01 &  4.38350e-01 & 7.95510e-01 \\ \hline
		10  	& 3.00631e-07 &  3.34978e-05 &  3.00285e-01 & 3.76380e+00 \\ \hline
		20  	& 1.10663e-18 &  1.19494e-15 &  8.57536e+00 & 2.00083e+02 \\ \hline
		50  	& 3.47335e-63 &  4.80438e-57 &  6.60565e+05 & 8.65842e+07 \\ \hline
	\end{tabular}
\end{center}
\renewcommand{\arraystretch}{1}

W przypadku funkcji \eqref{fun1} i funkcji \eqref{fun2} obserwujemy znaczący wzrost dokładności wraz ze wzrostem ilości węzłów. 

Z kolei funkcje \eqref{fun3} i \eqref{fun4} nie interpolują się w sposób jakiego byśmy oczekiwali. Zwiększenie ilości węzłów nie powoduje zwiększenia dokładności. Wprost przeciwnie - zmniejsza ją. Efekt ten nazywa się efektem Runge'go. Polega on na tym, że dla pewnych funkcji błąd interpolacji na końach przedziału zwiększa się wraz ze wzrostem ilości węzłów.
Zaradzić można temu interpolując funkcję w węzłach które są przeskalowanymi zerami odpowiedniego wielomianu Czebyszewa, albo zastosować inny sposób interpolacji - interpolację funkcjami sklejanymi.

\begin{figure}[ht]
	\begin{center}
		\includegraphics[width=13cm]{lagrange_sin}
	\end{center}
	\caption{Interpolacja funkcji \eqref{fun1} w 3 równoodległych węzłach}
	\label{fig:rysunek1}
\end{figure}

\begin{figure}[ht]
	\centering
	\begin{subfigure}[ht]{0.5\textwidth}
		\includegraphics[width=\textwidth]{lagrange_c}
		\caption{Interpolacja funkcji \eqref{fun3} w 11 równoodległych węzłach}
		\label{fig:2}
	\end{subfigure}%
	\begin{subfigure}[ht]{0.5\textwidth}
		\includegraphics[width=\textwidth]{lagrange_d}
		\caption{Interpolacja funkcji \eqref{fun4} w 10 równoodległych węzłach}
		\label{fig:3}
	\end{subfigure}
\end{figure}





\renewcommand{\arraystretch}{1.5}  
\begin{center}
	\begin{tabular}{||c||c|c|c|c||} \hline
		\multicolumn{5}{||c||}{Tabela błędów dla N = 1000} \\ \hline
		Węzły Czebyszewa 	& \multicolumn{4}{|c||}{Funkcje} \\ \cline{2-5}
		& $\sin x$ & $e^x$ & $(x^{2}+1)^{-1}$ & $x/(x^{2} + \frac{1}{4})$ \\ \hline					
		3 		& 3.25042e-02 &  3.74479e+00 &  6.00596e-01 & 9.36666e-01 \\ \hline
		5 		& 9.78100e-04 &  1.85380e-01 &  4.02016e-01 & 8.29971e-01 \\ \hline
		10  	& 4.69702e-08 &  5.27884e-06 &  2.69159e-01 & 3.25376e-01 \\ \hline
		20  	& 6.38319e-21 &  6.96882e-18 &  3.75819e-02 & 8.38636e-02 \\ \hline
		50  	& 3.69287e-70 &  5.13405e-64 &  9.68221e-05 & 7.14083e-04 \\ \hline
	\end{tabular}
\end{center}
\renewcommand{\arraystretch}{1}

\begin{figure}[ht]
	\centering
	\begin{subfigure}[ht]{0.5\textwidth}
		\includegraphics[width=\textwidth]{chebyshev_lagrange_c}
		\caption{Interpolacja funkcji \eqref{fun3} w 11 węzłach Czebyszewa}
		\label{fig:2}
	\end{subfigure}%
	\begin{subfigure}[ht]{0.5\textwidth}
		\includegraphics[width=\textwidth]{chebyshev_lagrange_d}
		\caption{Interpolacja funkcji \eqref{fun4} w 10 węzłach Czebyszewa}
		\label{fig:3}
	\end{subfigure}
\end{figure}



Więcej obliczeń można znaleźć w pliku \textbf{program.ipynb}.

\section{Funkcje sklejane}


\begin{defi}
	\textbf{Funkcją sklejaną} S(x) stopnia s, określoną na przedziale $[a,b]$ nazywamy dowolną funkcję spełniającą warunki:
	
	\begin{itemize}
		\item w każdym przedziale $[t_i,t_{i+1}]$, gdzie $a = t_0 < t_i <...<t_n = b$, S jest wielomianem stopnia co najwyżej s.
		\item S oraz jej pochodne rzędu 1,2,...,s-1 są ciągłe na całym przedziale $[a,b]$.
	\end{itemize}
\end{defi}

\noindent Aby ujednoznacznić istnienie takiej funkcji n-tego stopnia wyznaczanej przez $n+1$ punktów $t_i$ definiuje się dodatkowe warunki:

\begin{defi}
	Funkcję sklejaną S(x) stopnia s, określoną na przedziale $[a,b]$
	nazywamy \textbf{naturalną} gdy:
	$S^{(s-1)}(a) = S^{(s-1)}(b) = 0$.
\end{defi}

Rozważmy interpolacje naturalną funkcją sklejaną III stopnia gdzie $t_i$ będą węzłami interpolacji.

\section{Interpolacja naturalną funkcją sklejaną III stopnia}

\begin{defi}
	Naturalna funkcja sklejana III stopnia S, interpolująca funkcję f w punktach $a = x_0 < x_i <...<x_n = b$ w przedziale $[a,b]$ spełnia warunki:
	
	\begin{itemize}
		\item w każdym przedziale $[x_i,x_{i+1}]$ S jest wielomianem stopnia co najwyżej 3.
		\item Dla wszystkich $x_i$ $S(x_i) = f(x_i)$.
	\end{itemize}
	
\end{defi}

\noindent Wprowadźmy oznaczenie:

\begin{defi}
	$M_k := S''(x_k)$
\end{defi}

\begin{twr}
	Wielkośći $M_k$ spełniają układ równań liniowych:\\
	$\lambda_k M_{k-1} + 2M_k + (1-\lambda_k) M_{k+1} = 6f[x_{k-1},x_k,x_{k+1}]$ \hfill $(k=0,1,...,n)$\\
	gdzie: $\lambda_k := h_k / ( h_k + h_{k+1})$, oraz $h_k = x_k - x_{k-1}$
\end{twr}
	
Układ równań z powyższego twierdzenia w postaci macierzowej ma trójprzekątniową macierz z dominującą przekątną. Można go więc rozwiązać w czasie liniowym.

\subsection{Algorytm rozwiązujący układ równań ze względu na $M_k$ w czasie linowym}

\begin{align}
	&q_0 := u_0 := 0 \\
	\notag
	&\begin{rcases}
		p_k := \lambda_k q_{k-1} + 2 \\ \notag
		q_k := (\lambda_k - 1) / p_k \\ \notag
		u_k := (d_k -\lambda_k u_{k-1}) / p_k
	\end{rcases}
	\text{$(k=1,2,...,n-1)$}
\end{align}
gdzie
\begin{equation*}
	d_k = 6 f[x_{k-1},x_k,x{k+1}] \textnormal{ } (k=1,2,...,n-1)
\end{equation*}
Wówczas
\begin{align*}
	&M_{n-1} = u_{n-1} \\ \notag
	&M_k = u_k + q_k M_{k+1} \textnormal{ } (k =n-2,n-3,...,1 )
\end{align*}
\\

\noindent Naturalna funkcja sklejana III stopnia interpolująca funkcję f w przedziale $[a,b]$ i punktach $a = x_0 < x_i <...<x_n = b$ dana jest wzorem:

\begin{align}
	S(x) = h_{k}^{-1}&[\frac{1}{6} M_{k-1}(x_k - x)^3  
					 +\frac{1}{6} M_k (x - x_{k-1})^3 \\
					 \notag
					 &+(f(x_{k-1}) - \frac{1}{6} M_{k-1} h_{k}^2) (x_k -x ) \\
					 \notag
					 &+(f(x_{k}) - \frac{1}{6} M_{k} h_{k}^2) (x_{k-1} -x )]
\end{align}


\section{Testy}
Po zaimplementowaniu algorytmu znajdującego naturalną funkcję sklejaną III stopnia dla danej funkcji interpolowanej wykonałem testy dla funkcji\eqref{fun1} \eqref{fun2} \eqref{fun3} \eqref{fun4}.

\renewcommand{\arraystretch}{1.5}  
\begin{center}
	\begin{tabular}{||c||c|c|c|c||} \hline
		\multicolumn{5}{||c||}{Tabela błędów dla N = 1000} \\ \hline
		Węzły równoodległe 	& \multicolumn{4}{|c||}{Funkcje} \\ \cline{2-5}
		& $\sin x$ & $e^x$ & $(x^{2}+1)^{-1}$ & $x/(x^{2} + \frac{1}{4})$ \\ \hline					
		3 		& 2.00170e-02 &  7.82994e+00 &  6.01194e-01 & 9.51746e-01 \\ \hline
		5 		& 1.06607e-03 &  2.41584e+00 &  2.79309e-01 & 7.78300e-01 \\ \hline
		10  	& 3.98466e-05 &  5.17255e-01 &  1.42857e-01 & 1.27153e-01 \\ \hline
		20  	& 1.95842e-06 &  1.18143e-01 &  1.23295e-02 & 5.81372e-03 \\ \hline
		50  	& 4.39481e-08 &  1.78329e-02 &  1.47830e-04 & 6.99450e-04 \\ \hline
	\end{tabular}
\end{center}
\renewcommand{\arraystretch}{1}

\renewcommand{\arraystretch}{1.5}  
\begin{center}
	\begin{tabular}{||c||c|c|c|c||} \hline
		\multicolumn{5}{||c||}{Tabela błędów dla N = 1000} \\ \hline
		Węzły Czebyszewa 	& \multicolumn{4}{|c||}{Funkcje} \\ \cline{2-5}
		& $\sin x$ & $e^x$ & $(x^{2}+1)^{-1}$ & $x/(x^{2} + \frac{1}{4})$ \\ \hline					
		3 		& 2.67947e-02 &  4.78534e+00 &  5.52584e-01 & 9.51746e-01 \\ \hline
		5 		& 1.74205e-03 &  1.30820e+00 &  2.79309e-01 & 7.78300e-01 \\ \hline
		10  	& 1.74219e-04 &  1.06546e-01 &  2.84800e-01 & 2.11489e-01 \\ \hline
		20  	& 9.92148e-06 &  7.03993e-03 &  5.31571e-02 & 3.32994e-02 \\ \hline
		50  	& 2.47689e-07 &  1.72779e-04 &  1.15453e-03 & 3.89304e-03 \\ \hline
	\end{tabular}
\end{center}
\renewcommand{\arraystretch}{1}

\begin{figure}[ht]
	\begin{center}
		\includegraphics[width=13cm]{spline_c}
	\end{center}
	\caption{Interpolacja funkcji \eqref{fun3} w 10 węzłach}
	\label{fig:rysunek3}
\end{figure}

\begin{figure}[ht]
	\begin{center}
		\includegraphics[width=13cm]{spline_d}
	\end{center}
	\caption{Interpolacja funkcji \eqref{fun4} w 11 węzłach}
	\label{fig:rysunek4}
\end{figure}


Wyeliminowaliśmy efekt Runge'go w dwóch ostatnich funkcjach, jednakże funkcje sklejane niezbyt dokładnie odwzorowują funkcję. Mimo wielokrotnego zwiększania ilości węzłów, błąd maleje niewiele. W przypadku dwóch pierwszych funkcji lepiej sprawdzała się interpolacja wielomianowa.
	
	
\section{Wnioski}

W przypadku niektórych funkcji, interpolacja wielomianowa nie sprawdza się. Błąd interpolacji na końcach przedziałów rośnie wraz ze wzrostem ilości węzłów. Warto zastosować wtedy interpolacje funkcjami sklejanymi. Efekt Runge'go zostanie wtedy wyeliminowany. 

Funkcje sklejane dobrze odwzorowują ogólny kształt funkcji, jednak ich dokładność wraz ze zwiększaniem ilości węzłów rośnie bardzo powoli. W szczególności dużo wolniej niż dla interpolacji wielomianowej gdy efekt Runge'go nie występuje. Zatem gdy dla danej funkcji błąd interpolacji nie rośnie na końcach przedziałów dobrze jest interpolować wielomianem, w przeciwnym wypadku funkcją sklejaną.


\begin{thebibliography}{9}
	\itemsep2pt
			
	\bibitem{kincaid} David Kincaid, Ward Cheney - "Analiza Numeryczna"
	
	\bibitem{prezentacja} \url{https://www.math.ntnu.no/emner/TMA4215/2008h/cubicsplines.pdf}
	
	\bibitem{wolfram_mathworld} Weisstein, Eric W. "Cubic Spline." From MathWorld--A Wolfram Web Resource. \url{http://mathworld.wolfram.com/CubicSpline.html}
	
	\bibitem{wiki} \url{https://en.wikipedia.org/wiki/Spline_interpolation}
	
	\bibitem{wikipedia_pl}
	\url{https://pl.wikipedia.org/wiki/Interpolacja_(matematyka)}
	
	\bibitem{runge}
	\url{https://en.wikipedia.org/wiki/Runge%27s_phenomenon}
	 
\end{thebibliography}

\end{document}