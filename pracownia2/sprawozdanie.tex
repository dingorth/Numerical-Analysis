\documentclass{article}
\usepackage[utf8]{inputenc}
\usepackage{polski}

\usepackage{bbm}
\usepackage{graphicx}    
\usepackage{caption}
\usepackage{subcaption}
\usepackage{epstopdf}
\usepackage{amsmath,amssymb,amsfonts,amsthm,mathtools}
\usepackage{hyperref}
\usepackage{url}
\usepackage{comment}
\usepackage[section]{placeins}
\newtheorem{defi}{Definicja}
\newtheorem{twr}{Twierdzenie}


\author{Jan Mazur 281141}
\date{Wrocław, \today}
\title{\textbf{Naturalne funkcje sklejane III stopnia} \\ Sprawozdanie do zadania P.2.9	}

\begin{document}
\maketitle

\section{Wstęp}
Interpolacja to ...

Może ustalmy normę dla naszych obliczeń. To norma z zadania.

\section{Interpolacja wielomianowa}
Zinterpolujmy funkcję w n+1 punktach wielomianem n-tego stopnia.
Zastosujemy algorytm znajdjący wielomian interpolacyjny w postaci Newtona.

Funkcja Runge'go
Całkiem fajnie działa. Psuje się na końcach przedziałów. Czasem im więcej punktów tym większy błąd - większa norma. To bardzo źle.

\section{Funkcje sklejane}

Cała ta teoria. Stopnie funkcji. Okresowa, naturalna.
Zajmiemy się naturalną.

\section{Interpolacja naturalną funkcją sklejaną III stopnia}
Macierzowy układ równań. Trójprzekątniowa macierz z dominującą przekątną.
Momenty - drugie pochodne.
Algorytm rozwiązujący w czasie liniowym.

\section{Testy}
Wykresiki i liczenie błędów.

Wybór punktów interpolacyjnych.
	
	
\section{Wnioski}
Jeśli zwykła interpolacja bardzo odstaje w niektórych miejscach to lepiej interpolować splinem.

Czy punkty równoodległe?


\begin{thebibliography}{9}
	\itemsep2pt
			
	\bibitem{kincaid} David Kincaid, Ward Cheney - "Analiza Numeryczna"
	
	\bibitem{prezentacja} \url{https://www.math.ntnu.no/emner/TMA4215/2008h/cubicsplines.pdf}
	
	\bibitem{wolfram_mathworld} Weisstein, Eric W. "Cubic Spline." From MathWorld--A Wolfram Web Resource. \url{http://mathworld.wolfram.com/CubicSpline.html}
	
	\bibitem{wiki} \url{https://en.wikipedia.org/wiki/Spline_interpolation}
	 
\end{thebibliography}

\end{document}